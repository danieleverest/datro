%% Generated by Sphinx.
\def\sphinxdocclass{report}
\documentclass[letterpaper,10pt,openany,oneside,english]{sphinxmanual}
\ifdefined\pdfpxdimen
   \let\sphinxpxdimen\pdfpxdimen\else\newdimen\sphinxpxdimen
\fi \sphinxpxdimen=.75bp\relax

\PassOptionsToPackage{warn}{textcomp}
\usepackage[utf8]{inputenc}
\ifdefined\DeclareUnicodeCharacter
 \ifdefined\DeclareUnicodeCharacterAsOptional
  \DeclareUnicodeCharacter{"00A0}{\nobreakspace}
  \DeclareUnicodeCharacter{"2500}{\sphinxunichar{2500}}
  \DeclareUnicodeCharacter{"2502}{\sphinxunichar{2502}}
  \DeclareUnicodeCharacter{"2514}{\sphinxunichar{2514}}
  \DeclareUnicodeCharacter{"251C}{\sphinxunichar{251C}}
  \DeclareUnicodeCharacter{"2572}{\textbackslash}
 \else
  \DeclareUnicodeCharacter{00A0}{\nobreakspace}
  \DeclareUnicodeCharacter{2500}{\sphinxunichar{2500}}
  \DeclareUnicodeCharacter{2502}{\sphinxunichar{2502}}
  \DeclareUnicodeCharacter{2514}{\sphinxunichar{2514}}
  \DeclareUnicodeCharacter{251C}{\sphinxunichar{251C}}
  \DeclareUnicodeCharacter{2572}{\textbackslash}
 \fi
\fi
\usepackage{cmap}
\usepackage[T1]{fontenc}
\usepackage{amsmath,amssymb,amstext}
\usepackage[english]{babel}
\usepackage{times}
\usepackage[Bjarne]{fncychap}
\usepackage{sphinx}

\usepackage{geometry}

% Include hyperref last.
\usepackage{hyperref}
% Fix anchor placement for figures with captions.
\usepackage{hypcap}% it must be loaded after hyperref.
% Set up styles of URL: it should be placed after hyperref.
\urlstyle{same}

\addto\captionsenglish{\renewcommand{\figurename}{Fig.}}
\addto\captionsenglish{\renewcommand{\tablename}{Table}}
\addto\captionsenglish{\renewcommand{\literalblockname}{Listing}}

\addto\captionsenglish{\renewcommand{\literalblockcontinuedname}{continued from previous page}}
\addto\captionsenglish{\renewcommand{\literalblockcontinuesname}{continues on next page}}

\addto\extrasenglish{\def\pageautorefname{page}}

\setcounter{tocdepth}{1}



\title{Operation Schindlers Ark}
\date{Oct 15, 2014}
\release{0.1.0}
\author{Author(s): Sion Buckler}
\newcommand{\sphinxlogo}{\vbox{}}
\renewcommand{\releasename}{Release}
\makeindex

\begin{document}

\maketitle
\sphinxtableofcontents
\phantomsection\label{\detokenize{index::doc}}



\chapter{Release Notes and Notices}
\label{\detokenize{releasenotes:release-notes-and-notices}}\label{\detokenize{releasenotes::doc}}
This section provides information about what is new or changed, including urgent issues and documentation updates.


\section{Version 0.1.0}
\label{\detokenize{releasenotes:version-0-1-0}}
This is the first release/ draft of this technical document.


\subsection{Older Versions}
\label{\detokenize{releasenotes:older-versions}}
The table below contains information and links to, older versions of this document.


\begin{savenotes}\sphinxattablestart
\centering
\sphinxcapstartof{table}
\sphinxcaption{Table 1.0 - Older Versions of this Document}\label{\detokenize{releasenotes:id1}}
\sphinxaftercaption
\begin{tabular}[t]{|\X{25}{100}|\X{25}{100}|\X{25}{100}|\X{25}{100}|}
\hline
\sphinxstyletheadfamily 
archive date
&\sphinxstyletheadfamily 
version
&\sphinxstyletheadfamily 
description
&\sphinxstyletheadfamily 
download link
\\
\hline
YYYY-MM-DD
&
0.x.x
&
N/A
&
no older version
\\
\hline
\end{tabular}
\par
\sphinxattableend\end{savenotes}


\subsection{Version 0.0.0}
\label{\detokenize{releasenotes:version-0-0-0}}
N/A


\section{Known and Corrected Issues}
\label{\detokenize{releasenotes:known-and-corrected-issues}}
Below is a table of pending issues which have been reported to our team.
These issues will be cleared from this list as and when they are remedied.


\begin{savenotes}\sphinxattablestart
\centering
\sphinxcapstartof{table}
\sphinxcaption{Table 1.1 - Known Issues}\label{\detokenize{releasenotes:id2}}
\sphinxaftercaption
\begin{tabular}[t]{|\X{10}{100}|\X{10}{100}|\X{20}{100}|\X{60}{100}|}
\hline
\sphinxstyletheadfamily 
date
&\sphinxstyletheadfamily 
version
&\sphinxstyletheadfamily 
subject
&\sphinxstyletheadfamily 
description
\\
\hline
YYYY-MM-DD
&
0.1.0
&
Draft
&
first draft only
\\
\hline
\end{tabular}
\par
\sphinxattableend\end{savenotes}

\sphinxstylestrong{Comments} - none


\section{Recently Updated Topics}
\label{\detokenize{releasenotes:recently-updated-topics}}
Nothing significant to report


\chapter{Introduction}
\label{\detokenize{introduction:introduction}}\label{\detokenize{introduction::doc}}
“MAN CEASES TO BE THE SLAVE AND TOOL OF HIS ENVIRONMENT AND CONVERTS HIMSELF INTO THE ARCHITECT OF HIS OWN DESTINY”, \sphinxstylestrong{CHE GUEVARA}

\sphinxstylestrong{Much of the framework of the fifth domain e.g. cyberspace, delineates from the commons of the fourth domain: land. Therefore looking backwards on the history of land, can prove invaluable when looking forward in the construction of cyberspace.}

More than three thousand years ago, after banishment from his birth land, Brutus of Troy formed an army of exiled and formerly enslaved Trojans. After many battles, General Brutus led an expedition, bringing his people to the shores of Albion. They defeated the islands giants, claimed the land as their own and built a city. Albion later renamed to ‘Britain’ and the city ‘London’. Brutus was the first King of Britain and enjoyed twenty years in power.

Here in the twenty first century, a platoon-sized community of this Kings descendants, serving under Her Majesty Queen Elizabeth II (Defender of the Faith and Commander and Chief of the British Armed Forces), have set out on an equally bold new expedition, into cyberspace. The group identifies itself as ‘the Extropy Trust’, a word deriving from a paradigm called ‘Extropism’ (an evolving framework of values and standards for continuously evolving the human condition (by applying the ‘proactionary principle’).

These fifty subjects of the British monarch, primarily comprise of ex and activeduty, senior non-commissioned soldiers and police, specialising in many aspects of computer forensics, telecommunications, cyber security and electronic warfare. Their unique skills combined with their shared values and standards (based on King Johns thirteenth century Magna Carta Charter) resulted in Operation Tech Saxon in Dubai in 2012 and the Trusts first charter: ‘internet access as a free and inherent right’.

By 2014, after much tribulation, Sir Tim Bernes-Lee (the British inventor/ King of the fifth domain) revealed how he had adopted this new paradigm and charter: calling on the World Wide Web to crowd-source a new ‘internet’ Magna Carta Charter/ Bill of Rights. The Queen also began work on revising the original to produce a Magna Carta Charter II, which is now in draft. The unprecedented success of the Trust’s first charter become increasingly evident, when the velocity of the economic opportunity it unveiled, was fully realised. According to a new Deloitte report, free internet access, as a human right, could increase global productivity by as much as 25\% (a GDP of \$2.2 trillion), creating 140+ million new jobs and lifting 160+ million people out of poverty. The extropism paradigm is clearly a force for good: warranting this societies newly proclaimed title - the ‘United Technocracy of Extropy’ (UTE).

During the UTE’s expedition, it also became increasingly evident that anonymity protects democracy: permitting the necessary level of free speech offences to take place, in order to provoke a paradigm shift and evolve society and the economy. Trust in human qualities (even those of anonymous people, via the technologies that enable this) are advantageous when defending a society and its paradigm from intelligent risks and threats. These human qualities resonate in the British Armed Forces and the UTE’s internet community, which resulted in the ‘Tech Saxon’ proposal and Charter, to the United Nations International Telecommunications Union, in Dubai in 2012. The UTE’s first charter was also a consequence of liberation from (not employment of) legal agreements, entities, titles and other archaic, fictional instruments that society, not humanity, have always looked upon for social and economic order. Until now.

The drink in the hand of any citizen, of any regimes political party, is no longer Champaign! In the information age, social and economic architecture as we know it (which has served the British Empire so well, for thousands of years) is turning out to be a long standing, hybrid form of a doctrine, identified as ‘Statism’. A paradigm, which has clearly outlived its usefulness in the 21st century. Frivolous offences, which now includes the sharing of information, makes society’s method of governance, in many ways; more restricted and harmful to its people than ancient civilisation. This method of governance has become a barbaric submissions to bullying or the kind of blind trust in its instruments; one would expect to see of cult follower as they receive instruction to drink a Champaign-labelled cup of cyanide. Secondly, the Statism paradigm has such terrorising and oppressive tentacles (which now extend so deep into the communities, homes, minds, and behaviours of its citizens) that freedom of thought and expression and freedom to exercise a sense of constitution i.e. challenge and offend, is near enough absent in modern people.

Civil obedience now appears to be the result of a ‘closing-wall’ ransom tactic on claustrophobics, rather than lawful governance and an exchange between two consenting entities. The relationship between the state and its citizen is now a fear driven surrender of property, produce, children, self, liberties and even ones own thoughts. Citizenship to the state has become surrender to the Statism doctrine. Welcoming into our every day life: the centrally controlled monetary system, the jurisdiction of statute law and not forgetting, the afforded privilege of being able to communicate digitally.

How long will UK citizens favour this form of governance when they are still, unbeknown too many, within their inherent legal and lawful right, to renounce citizenship at an individual level and revert to the commons as subjects of their country’s monarch? As an example of what is being described above, the UK state has now claimed so much of everything British (including 99.7\% of land and people) that British police are now officers of the UK, British soldiers are now ‘troops’ of the UK, the British justice system are secret kangaroo courts of the UK. Even Britain’s common language has been ‘totalitarianised’ into UK ‘legalise’. The British people must call upon demand and reclaim their commons back from the state: liberating their land and their resources from the exploitation of the state, its fictional instruments and misapplication of words, which by very design are attempts to deceive.

A starting point of practice of the extropism paradigm is to reject use of legalise to mislead and deceive: one such example includes correcting anyone distorting and mixing up reference to UK society and citizenship, and Britain the country/free British people, in day-to-day conversation. They are not the same thing and one should not pollute, or give credence to the other. This is all, absolutely necessary, for British people to once again understand and enjoy, free exercise of their inherent civil liberties, affairs and incur fuller benefits of labour and ones own produce. Not as false society would simulate, but in actual reality. Without this reclaiming of the commons, the British people will continue their descent, into the UK’s disparity and subjugation framework. British people still have the opportunity to be architects of their society, without leaving it to the state. While renouncing from the state appears to be, and remains an option, the UK has in fact, closed that door behind the on looking British people. To turn around now would be to acknowledge that there is little remaining outside of the all-powerful state, including the commons (of the commonwealth).

The British people absolutely must re-develop their commons, in order to manage the social and economic risks and threat, associate with this precarious situation we find ourselves. This includes operating as community projects and private trusts, using virtual currency with anonymity/ equitable title and not being ashamed to declare these sovereign under the monarch and outside of the contract with the state and bank, e.g. none of their business (quiet literally).

In many ways, this movement is already happening in cyberspace, but until now, its framework has been somewhat of a novelty, and its been done via the simulation (afforded, revocable privilege) of a right to internet access, which is what makes the fifth domain such a playground of social, democratic and sovereign phenomenons. Internet as a free and inherent right truly ensures this domain becomes a safe haven for humankind. In contrast to the endless freedom and empowerment cyberspace brings, increasing numbers of people have begun experiencing a personal revolution and are questioning their less liberating capacity as a citizen, not to mention their fate if they permit the ever-growing disparity, deceit and repression of the state and banks, to continue. While the banks have a stranglehold on the majority of peoples surrounding environment and local economy, the state makes ransoms on the social mood and moral conscience of a nation: reducing an entire country and its people, to that of an egg farm and its hens. The minority that speak out, and the people who follow in their path, will look to the internet for remedy to the chronic humanitarian issues they identify. We are at a point in time where a cataclysmic, if not medium scale event, could easily transform this minority of ‘truth seekers, ‘alternates’ and other such offenders of the status quo: into a flock of terrified people. Who, instead of sharing the fate of an archaic, collapsing paradigm and social/ economic framework, seek liberation from it. Alarm bells should already be ringing now the UK state is coming for British people who speak out, which begs the question: who will remain when the state come for you or I if we don’t say something right now?

The cyber social and economic alternatives to the monopoly offer of the state and bank cartels are still somewhat of a playground of social networks and virtual currencies. It is time to unite both of these, form an alternative socio-economic framework and defend it, as part of the commons under the commonwealth realm and Her Majesty the Queen. The King of the internets Magna Carta Charter and the Queens Magna Carta Charter II affirms the importance of a move in this direction. The UTE delineated its first charter from this extropism paradigm, putting it into practice its free internet solution and other equally liberating new technologies. It is now ready to lead others in this transcendence to the truth, warmth and the peace paradigm of the information age and help others abandon the cold, sterile, conforming war paradigm of the post-industrial age. In order to do this, statism must be studied, and reverse-engineered, while extropism further explored and the relationships between these two identified, in order to transition from this collapsing paradigm to the rising one, as smoothly as possible.

The varying forms of Magna Carta will prove an invaluable bridge in this paradigm shift, which is necessary to graduate to the next level of human development, as was true of every one of the previous social and economic revolutions: hunter-gatherer to nomad, nomad to agricultural, agricultural to industrial. The information age generation, unilaterally concur that the statism paradigm (and its fictional, paper-based-instrument framework), is an exploitation and disparity system, belonging to a select few. The United Kingdom (society) which occupies Britain (the country) could not be further from a United Kingdom. 99.7\% of the population of Britain, (15\% of which are immigrants) inhabit the island in a largely avoidable and unnecessary disparity caused by the statism paradigm, imprisoning them and completely disconnecting them from the 0.3\% elite: the only ones truly living how each descendant (and inheritor) of King Brutus’ island, should and could live. In reality, this minority have no lawful right to afford the masses their share of this common inheritance: it is instead, consensually surrendered, using the abovementioned mechanisms of deceit. Birth rights and fair dividend of inherited earth (and the resources upon it), must now be firmly asserted, and claimed back! There is no greater power-levelling forum for this paradigm shift and revolution than cyberspace.

Extropism is not just the awakening from the statism doctrine or vigilance against its powers; it is actually a viable alternative and path forward. The founding Extropians are by very trade, keepers of peace, defenders of democracy and constitution and most importantly fair and good British people. In 2012 the UTE challenged the statism paradigm and framework and endured the usual traits of its victims, attempted-discretisation, purposeful obstructions etc. Despite these challenges, the charter made its way into the hands of shielded world leaders, who’d attended this secret closed-off event, tenacious enough to call itself a democratic ‘open debate’. The charter to “transform digital communications: from an affordable privilege, to a ‘free and inherent right” conflicted with the events very evident political agenda: to prolong the affording of internet access as a privilege. However, the UTE’s assertion of right and unforgiving, direct (in the flesh) appeal to the humanity of the people attending the event, (a principle of the extropian paradigm) resulted in it finally being considered, and debated. It is now a widely accepted milestone in human development and its unmatched economic impact is hugely celebrated. However, the implications of this charter on personal liberties have remained largely unexplained and unexplored until the disclosure of this paper.

Similar to the beginning of Britain, the United Technocracy of Extropy (UTE) finds itself sharing the discovery of this expedition, with giants e.g. Facebook, Google, the UK etc. The UTE is clearly David in a world of Goliaths, but since 325 million people in 2017 (and 50+ million by 2016) will be users of Waves free internet solution, it is inevitable that a percentage will renounce their assets, as well as themselves and move to the UTE’s electoral. The UTE’s foresight and agility in size and skills, has evidently been advantageous: with a mere \$1 million investment to date, this organisation has developed a proof of concept of a free internet technology solution called WAVE: boasting thousands of pre-registrations each month and, an already estimated worth of around \$17+ million. WAVE features the world’s first Private, Virtual Currency Securities Exchange (PVCSE). This innovative value exchange platform has already received hundreds of pre-launch requests to trade with a maximum seller seeking \$40k @ \$0.01102 a share, and a minimum investment offer of \$115k @ \$0.0096c a share (of the one billion total shares in circulation). The PVCSE will launch months ahead of the NASDAQ’s version, and will host the remainder of Wave’s \$2.3 million, Series A investment round: financing the 2015 public beta release, a dedicated campus for our newly uniformed engineers and security team, and this newly proposed operation for the UTE.


\chapter{Social Framework as a Service (SFaaS)}
\label{\detokenize{sfaas:social-framework-as-a-service-sfaas}}\label{\detokenize{sfaas::doc}}
A secondary application of this technology is its open-sourced militarisation into, what is being referred to as the worlds first ‘Social Framework as a Service’ (SFaaS). To elaborate on what this is, first consider Private Military Contractors (PMC’s) which is an explosive industry in which the UTE now associates itself. PMC’s accounted for a seventh of the forces on the ground in Iraq in 2003. By 2007 these soldiers-for-hire outnumbered the ones Governments sent. In 2014 this relatively unregulated industry grew to \$218 billion and continues to grow at an exponential rate. This new proposal from the UTE is the cyber equivalent of a PMC, except the technology will not empower corporations in exchange for relinquishment of their capital. Instead, the Wave technology will liberate free internet users in unstable regions with SFaaS, with the inevitable expectancy of them relinquishing their ‘consent to be governed’ to the UTE.

At the heart of the internet and cyberspace is the main nomenclatures of cloud computing: platform as a service (PaaS), infrastructure as a service (IaaS) and the list goes on (DaaS, BaaS and ITMaaS). Cities are also becoming smarter and more interconnected with the ‘internet of things’ (IoT) which includes wearable devices. Online self-teaching platforms are already so popular; they are now being favoured over state-run educational institutions. We are seeing the same trends with virtual currency over regular currency and crowd-sourced people-people apps, over car rental and hotel companies etc. It is just a matter of time before something as simple as a social, dispute resolution platform emerges and becomes more popular than the state services of a court. We can even expect to see a social debt cancellation circle, which uses the shrinking ‘degrees of separation’ to add liquidity back into the economy. SFaaS brings all of these together in one user interface, making it the inevitable next nomenclature in cloud computing.

The features necessary for SFaaS are already in place with Wave, subject to some re-configurations. For example, land, vehicle, business and even property registration with the state will evolve, replacing deeds with a kind of social network page, which could also access our PVCSE, giving owners the ability to trade/share ownership and distribute royalties accordingly. These newly renounced items from the state will fall onto common ground under the monarch, before they are consensually surrender into the logic of this technocracy by the equitable owners. This new society will organically form from this migration of technologies, resources and eventually people. One of the greatest benefits of life under the UTE in Britain will be the restoration of balance between capitalism/statutes, and socialism and the commons - using innovative new technologies.

Each registration of land, resources (including the electoral of human resources e.g. people) and technology, will hold commercial value via the PVCSE. However, a new logarithmic social endorsement system (the equivalent of an open credit scoring system) will introduce a new, social bearing on its perceived value to society, as oppose to just its economic value. For example, in one trade union letter during the industrial revolution, it stated that 170 machines extended 17 miles across Leeds and outputted an unprecedented amount of produce and subsequent profit for the company and its shareholders. However, there was an appeal to the humanity and mercy of these giants, as each machine replaced ten workers, driving an entire community of homes, businesses and schools from their roots, to make way for the nation’s economic need for a largely derelict, sterile industrial estate.


\chapter{Democracy in Technocracy}
\label{\detokenize{democracy-technocracy:democracy-in-technocracy}}\label{\detokenize{democracy-technocracy::doc}}
Within Wave, the logic gates and algorithms that determine what each free internet user can and cannot do are determined via the convention of a support tickets and feedback system. The alpha release of Wave and launch of the PVCSE in Q4 2014, will now be closely followed by a new voting feature, which will democratise the direction of the development of the platform by giving its anonymous beneficiaries real-time enjoyment of their voting rights. With SFaaS these ‘logic-gate laws’ will come into effect in much the same way, and are already interlinked with each user profile giving the UTE, SFaaS and Wave a similar multi level decision making platform as UK Parliament. The only differences being, Instead of a house of commons, free internet users can vote on Wave and extropians can vote on the UTE’s SFaaS. Instead of lords, the founding extropians and their successors will vote on the direction of the UTE as well as SFaaS and Wave.


\chapter{Schindlers Ark}
\label{\detokenize{schindlers-ark:schindlers-ark}}\label{\detokenize{schindlers-ark::doc}}
SFaaS would be of immense value for unstable regions: and who better to implement this innovative solution, than the UTE’s community of founding, military, police and PMC extropians. However, in stable regions and developed countries, there would need to be a cataclysmic event (or as explained earlier maybe just a medium scale event) before such a solution would gain adoption. There is however, a cataclysmic event rumoured to occur in Britain sometime during the lifespan of this operation. If Wave were providing free internet in Britain and the UTE have SFaaS ready to deploy to this network, it could come into power: re-introducing stability by acting as a viable socio-economic value exchange/ resource-management and self-governance solution.

This is the unexplained and largely unexplored fulfilment of the charter’s promise, to liberate and empower people. Overcoming the disparity of a framework of legal fictions and instruments, which has superseded real people and technologies; architectural flaws of a post-industrial society and economic framework, which is no match for the knowledge-based ages colonising and preferred alternative solutions.

Cyprus is a great example and a clue about how Governments, (including the UK), might behave when they are in receivership e.g. bankruptcy. Secret plans are already rumoured to be in motion, to freeze UK citizens banking privileges, restrict daily withdraw to as little as 100 GBP per person and convert peoples savings into government bonds with no warnings or notice. This is will be a clear attempt by the elite’s bankers and politicians to shift the UK’s debts onto the 99.7\%, their children and their children’s children. This is a reoccurring tend throughout history which keeps the disparity going, generation to generation. The UTE can actually benefit from this inevitable destabilisation thanks to technologies such as PVCSE, SFaaS and Wave.


\chapter{The Movement}
\label{\detokenize{movement:the-movement}}\label{\detokenize{movement::doc}}
The Dominican Republic (DR) is a western-influenced Caribbean island, which is surrounded by British Oversease Territories, which are members of the commonwealth realm, with the British monarch; Queen Elizabeth II at its honouree head. The DR has developed a reasonable infrastructure and hundreds of high footfall holiday resorts, many five star. Yet less than 15\% of homes on the island have internet access, making this an ideal, high-yield location for the alpha release trials of Wave.

Almost 70\% of the DR’s police and military, protect commercial interests of elites (over the needs of their own people), even more so than in Britain, while no cyber defence force exists. This non-UN terrain also serves as an ideal training ground and base of operation for the UTE to foment its revolutionary SFaaS system. While the Wave team focus on free internet access as UTE’s ‘front line’ engineers (gaining new ground), a second team of engineers will work to roll out SFaaS onto this network and gain the consent of its users, to be governed by the UTE.


\chapter{Summary}
\label{\detokenize{summary:summary}}\label{\detokenize{summary::doc}}
The ‘United Technocracy of Extropy’ (UTE) is a force for good and accepts the responsibility of challenging an old paradigm to make way for a new. In the aftermath of disparity of the statism paradigm, extropians (through the gateway of internet access as a free, inalienable right): will use advanced technologies and methods to de-centralise and logarithmically socially score peoples resources on their land in a completely new social and economic framework. Making life here on earth, pure Extropy.

Cyberspace is our playground for this revolution, so let us play!


\chapter{\sphinxstylestrong{Document Author(s):}}
\label{\detokenize{index:document-author-s}}
\sphinxstylestrong{Sion Buckler}



\renewcommand{\indexname}{Index}
\printindex
\end{document}